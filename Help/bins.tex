\documentclass[aps,pra,preprint]{revtex4-2}
\usepackage{amsmath}
\usepackage{amsfonts}
\usepackage{amsthm}
\usepackage{graphicx}
\usepackage[dvipsnames,usenames]{color}
\usepackage[pdftex,bookmarks=false,pdfstartview=FitH,colorlinks]{hyperref}
\hypersetup{linkcolor=Sepia,citecolor=Sepia}
\usepackage[all]{hypcap}

\theoremstyle{definition}
\newtheorem{theorem}{Theorem}%[section]
\newtheorem{proposition}[theorem]{Proposition}
\newtheorem{result}[theorem]{Result}
\newtheorem{definition}[theorem]{Definition}
\newtheorem*{example}{Example}
\newtheorem*{examples}{Examples}

\newcommand{\sub}[1]{_{\text{#1}}}
\newcommand{\unit}[1]{\,\mathrm{#1}} % attach units
\newcommand{\V}[1]{\mathbf{#1}} % bold vector or matrix
\newcommand{\grad}{\nabla} % gradient operator
\newcommand{\di}{\partial} % partial derivative
\newcommand{\kros}{\times} % vector product
\newcommand{\rom}[1]{\mathrm{#1}} % upright function name
\newcommand{\abs}[1]{\left\vert#1\right\vert} % generic absolute value
\newcommand{\set}[1]{\left\{#1\right\}} % put elements between { }
\newcommand{\Real}{\mathbb R} % Real numbers field
\newcommand{\Cmplx}{\mathbb C} % Complex numbers field
\newcommand{\eps}{\varepsilon} % variirtes epsilon
\newcommand{\To}{\longrightarrow} % a long right arrow
\newcommand{\Vomega}{\boldsymbol{\omega}}
\newcommand{\VOmega}{\boldsymbol{\Omega}}
\newcommand{\Veps}{\boldsymbol{\epsilon}}

\begin{document}

\title{Grain size bins}
\author{Naor Movshovitz}
\email{nmovshov@ucsc.edu}
\affiliation{UC Santa Cruz}

\begin{abstract}
Explains the size bin spacing and number density formulas.
\end{abstract}
\maketitle

According to \texttt{D'Allesio et al., 2001} the dust grain distribution is
$n(a)=n_0a^{-p}$, with $p$ a free parameter. Taking
$a_{\text{min}}=5\times10^{-9}\unit{m}$ and
$a_{\text{max}}=1\times10^{-3}\unit{m}$ and $p=3.5$. We take grains with radii
$a_i=a_02^{i/2}$, in bins $i=1\ldots{40}$. This translates to taking the number
density in the $i^{\text{th}}$ bin to be
\begin{equation}
n_i=n_0a_0^{-2.5}2^{\frac{-10i+7}{8}}(2^{1/4}-1).
\end{equation}

\end{document}
