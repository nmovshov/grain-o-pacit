\documentclass[aps,pra,preprint]{revtex4-2}
\usepackage{amsmath}
\usepackage{amsfonts}
\usepackage{amsthm}
\usepackage{graphicx}
\usepackage[dvipsnames,usenames]{color}
\usepackage[pdftex,bookmarks=false,pdfstartview=FitH,colorlinks]{hyperref}
\hypersetup{linkcolor=Sepia,citecolor=Sepia}
\usepackage[all]{hypcap}

\theoremstyle{definition}
\newtheorem{theorem}{Theorem}%[section]
\newtheorem{proposition}[theorem]{Proposition}
\newtheorem{result}[theorem]{Result}
\newtheorem{definition}[theorem]{Definition}
\newtheorem*{example}{Example}
\newtheorem*{examples}{Examples}
\DeclareMathOperator{\diff}{diff}
\newcommand{\sub}[1]{_{\text{#1}}}
\newcommand{\unit}[1]{\,\mathrm{#1}} % attach units
\newcommand{\V}[1]{\mathbf{#1}} % bold vector or matrix
\newcommand{\grad}{\nabla} % gradient operator
\newcommand{\di}{\partial} % partial derivative
\newcommand{\kros}{\times} % vector product
\newcommand{\rom}[1]{\mathrm{#1}} % upright function name
\newcommand{\abs}[1]{\left\vert#1\right\vert} % generic absolute value
\newcommand{\set}[1]{\left\{#1\right\}} % put elements between { }
\newcommand{\Real}{\mathbb R} % Real numbers field
\newcommand{\Cmplx}{\mathbb C} % Complex numbers field
\newcommand{\eps}{\varepsilon} % variirtes epsilon
\newcommand{\To}{\longrightarrow} % a long right arrow
\newcommand{\Vomega}{\boldsymbol{\omega}}
\newcommand{\VOmega}{\boldsymbol{\Omega}}
\newcommand{\Veps}{\boldsymbol{\epsilon}}

\begin{document}

\title{Over-simplified single-size model}
\author{Naor Movshovitz}
\email{nmovshov@ucsc.edu}
\affiliation{UC Santa Cruz}

\begin{abstract}
Notes to myself on the simple model. The purpose of this model is to estimate a
characteristic grain size in a steady state sedimentation scenario. Basically,
I try to find one grain size in each zone that equalizes the typical times of
sedimentation and coagulation.

\end{abstract}
\maketitle

\section*{Atmospheric data}
I read the gas density and temperature at 60 \emph{locations} in the atmosphere
from file, e.g., \textsc{atm.dat}. I convert this to 59 \emph{zones}, or
layers, with $l_k$ using the values in $Z_{k+1}$, $T_{k+1}$, and
$\rho^{\text{gas}}_{k+1}$. In other words, the gas density in the first zone is
taken from the \emph{second} line of \textsc{atm.dat}, and so on. (Should I use
mean values instead? Probably not.)

\section*{Grain independent properties}
I denote by $l$ the layer thickness, i.e., $l=\abs{\diff(Z)}$. The acceleration
of gravity at each zone is
\begin{equation}
g=\frac{GM_{\text{core}}}{Z^2},
\end{equation}
where $M_{\text{core}}=6.9\times10^{25}\unit{kg}$ is the core mass of the
protoplanet, and $G=6.673\times10^{-11}\unit{m^3\,kg^{-1}\,s^{-2}}$ is the
universal gravitational constant. The gas number density is
\begin{equation}
n_{\text{gas}}=\frac{\rho_{\text{gas}}N_\text{A}}{\mu},
\end{equation}
where $N_{\text{A}}=6.02\text{e23}\unit{\text{mole}^{-1}}$ is Avogadro's number
and $\mu=2.3\text{e-3}\unit{kg/\text{mole}}$ is the mean molecular weight of
the gas. The thermal velocity of the gas molecules is given by
\begin{equation}\label{eq:vthgas}
v_\text{th}^\text{gas}=\sqrt{\frac{8K_\text{B}T{N}_\text{A}}{\pi\mu}}.
\end{equation}
(Boltzmann's constant in mks is $K_\text{B}=1.38e-23\unit{J/^\circ{K}}$.)
Finally, the mean free path of a gas molecule is given by
\begin{equation}
\text{mfp}=\frac{1}{\sqrt{2}n_\text{gas}\sigma},
\end{equation}
with $\sigma=1\text{e-19}\unit{m^2}$ as the molecular collision cross section.
Table~\ref{tab:values1} gives some typical values of these dynamic properties.

\begin{table}
\caption{\label{tab:values1}Typical values of grain independent atmospheric and
dynamic properties.}
\begin{ruledtabular}
\begin{tabular}{ccccccc}
$Z\,[\rom{m}]$ & $T\,[\rom{^\circ{K}}]$ & $g\,[\rom{m/s^2}]$ &
$\rho_\text{gas}\,[\rom{kg/m^3}]$ & $n_\text{gas}\,[\rom{1/m^3}]$ &
$v_\text{th}^\text{gas}\,[\rom{m/s}]$ & $\text{mfp}_\text{gas}\,[\rom{m}]$ \\
\hline
$5.9\text{e9}$ & $154$ & $1.3\text{e-4}$ & $7.4\text{e-8}$ & $1.9\text{e19}$ &
$1.2\text{e3}$ & $3.6\text{e-1}$ \\ $2.1\text{e9}$ & $168$ & $1.0\text{e-3}$ &
$2.8\text{e-6}$ & $7.3\text{e20}$ & $1.2\text{e3}$ & $9.7\text{e-3}$ \\
$1.2\text{e9}$ & $257$ & $3.3\text{e-3}$ & $7.0\text{e-5}$ & $1.8\text{e22}$ &
$1.5\text{e3}$ & $3.8\text{e-4}$
\end{tabular}
\end{ruledtabular}
\end{table}

\section*{Sedimentation speed and time}
The Knudsen number is much greater than one in most of the zones and for most
conceivable grain sizes. The only exception is deep in the atmosphere, and even
then only for the largest grains. I therefore use Epstein drag in my
calculation. The drag force then is given by
\begin{equation}
F_\text{drag} = 
\frac{4\pi{a}^2\rho_\text{gas}v_\text{th}^\text{gas}v_\text{sed}}{3},
\end{equation}
and the sedimentation speed can be found by setting $F=mg$:
\begin{equation}
v_\text{sed} = 
\frac{\rho_\text{grain}g}{\rho_\text{gas}v_\text{th}^\text{gas}}\,a.
\end{equation}
The time to sediment out of the layer is $l/v_\text{sed}$. The sedimentation
time for a $10$ micron grain ranges from a few years in the uppermost layer to
hundreds of years deeper in the atmosphere.

\section*{Coagulation time}
In a steady state the total mass of dust coming into a layer per unit time is
equal to the total mass going out of the layer at the same time. There is
therefore a simple expression for the flux $F$ going through each layer,
\begin{equation}\label{eq:flux}
F=F_0\frac{Z_0^2}{Z^2},
\end{equation}
where $Z_0$ is the ``end'' of the atmosphere, and in each layer,
\begin{equation}
F=\frac{4\pi}{3}a^3\rho_\text{grain}n_\text{grain}v_\text{sed}.
\end{equation}
(At this time there is no source term of dust representing breakup of
planetesimals etc.) I take $F=4\text{e-12}\unit{kg\,m^{-2}\,s^{-1}}$ based on
$1\text{e-8}$ Earth masses a year falling onto the planet from the solar
nebula. I therefore have $n_\text{grain}$, the number density of grains in each
zone:
\begin{equation}
n_\text{grain}=\frac{3F}{4\pi{a}^3\rho_\text{grain}v_\text{sed}}.
\end{equation}
The mean distance between grains is
\begin{equation}
\text{mfp}_\text{grain}=\frac{1}{\sqrt{2}n_\text{grain}\sigma_\text{grain}}.
\end{equation}
For $\sigma$, the grains collision cross section, Morris takes $4\pi^2$,
because if the center of one grain gets closer than $2a$ to the center of
another grain they will collide. So
\begin{equation}
\text{mfp}_\text{grain}=\frac{1}{\sqrt{2}4\pi{a}^2n_\text{grain}}.
\end{equation}
The random speed of the grain due to Brownian motion is
\begin{equation}
v_\text{th}^\text{grain}=\sqrt{\frac{8K_\rom{B}T}{\pi{m}_\text{grain}}},
\end{equation}
and $t_\text{coag}=\text{mfp}_\text{grain}/v_\text{th}^\text{grain}$.

\section*{Characteristic grain size}
Setting $t_\text{sed}=t_\text{coag}$ and solving for $a$ yields
\begin{equation}
a = \left[\frac{3\sqrt{12K_\rom{B}T}}{\pi}
\frac{Fl\rho_\text{gas}^2(v_\text{th}^\text{gas})^2}
{\rho_\text{grain}^{7/2}g^2}
\right]^{2/9}.
\end{equation}
Substituting in Eqs.~\eqref{eq:flux} and~\eqref{eq:vthgas} we get
\begin{equation}
a = \left[
\frac{48\sqrt{3}N_\rom{A}(K_\rom{B}T)^{3/2}F_0Z_0^2\rho_\text{gas}^2l}
{\pi^2\mu\rho_\text{grain}^{7/2}g^2Z^2}
\right]^{2/9}.
\end{equation}

\end{document}
